% Options for packages loaded elsewhere
\PassOptionsToPackage{unicode}{hyperref}
\PassOptionsToPackage{hyphens}{url}
%
\documentclass[
]{article}
\title{Lab\_1}
\author{Brooke Wheeler}
\date{1/12/2022}

\usepackage{amsmath,amssymb}
\usepackage{lmodern}
\usepackage{iftex}
\ifPDFTeX
  \usepackage[T1]{fontenc}
  \usepackage[utf8]{inputenc}
  \usepackage{textcomp} % provide euro and other symbols
\else % if luatex or xetex
  \usepackage{unicode-math}
  \defaultfontfeatures{Scale=MatchLowercase}
  \defaultfontfeatures[\rmfamily]{Ligatures=TeX,Scale=1}
\fi
% Use upquote if available, for straight quotes in verbatim environments
\IfFileExists{upquote.sty}{\usepackage{upquote}}{}
\IfFileExists{microtype.sty}{% use microtype if available
  \usepackage[]{microtype}
  \UseMicrotypeSet[protrusion]{basicmath} % disable protrusion for tt fonts
}{}
\makeatletter
\@ifundefined{KOMAClassName}{% if non-KOMA class
  \IfFileExists{parskip.sty}{%
    \usepackage{parskip}
  }{% else
    \setlength{\parindent}{0pt}
    \setlength{\parskip}{6pt plus 2pt minus 1pt}}
}{% if KOMA class
  \KOMAoptions{parskip=half}}
\makeatother
\usepackage{xcolor}
\IfFileExists{xurl.sty}{\usepackage{xurl}}{} % add URL line breaks if available
\IfFileExists{bookmark.sty}{\usepackage{bookmark}}{\usepackage{hyperref}}
\hypersetup{
  pdftitle={Lab\_1},
  pdfauthor={Brooke Wheeler},
  hidelinks,
  pdfcreator={LaTeX via pandoc}}
\urlstyle{same} % disable monospaced font for URLs
\usepackage[margin=1in]{geometry}
\usepackage{color}
\usepackage{fancyvrb}
\newcommand{\VerbBar}{|}
\newcommand{\VERB}{\Verb[commandchars=\\\{\}]}
\DefineVerbatimEnvironment{Highlighting}{Verbatim}{commandchars=\\\{\}}
% Add ',fontsize=\small' for more characters per line
\usepackage{framed}
\definecolor{shadecolor}{RGB}{248,248,248}
\newenvironment{Shaded}{\begin{snugshade}}{\end{snugshade}}
\newcommand{\AlertTok}[1]{\textcolor[rgb]{0.94,0.16,0.16}{#1}}
\newcommand{\AnnotationTok}[1]{\textcolor[rgb]{0.56,0.35,0.01}{\textbf{\textit{#1}}}}
\newcommand{\AttributeTok}[1]{\textcolor[rgb]{0.77,0.63,0.00}{#1}}
\newcommand{\BaseNTok}[1]{\textcolor[rgb]{0.00,0.00,0.81}{#1}}
\newcommand{\BuiltInTok}[1]{#1}
\newcommand{\CharTok}[1]{\textcolor[rgb]{0.31,0.60,0.02}{#1}}
\newcommand{\CommentTok}[1]{\textcolor[rgb]{0.56,0.35,0.01}{\textit{#1}}}
\newcommand{\CommentVarTok}[1]{\textcolor[rgb]{0.56,0.35,0.01}{\textbf{\textit{#1}}}}
\newcommand{\ConstantTok}[1]{\textcolor[rgb]{0.00,0.00,0.00}{#1}}
\newcommand{\ControlFlowTok}[1]{\textcolor[rgb]{0.13,0.29,0.53}{\textbf{#1}}}
\newcommand{\DataTypeTok}[1]{\textcolor[rgb]{0.13,0.29,0.53}{#1}}
\newcommand{\DecValTok}[1]{\textcolor[rgb]{0.00,0.00,0.81}{#1}}
\newcommand{\DocumentationTok}[1]{\textcolor[rgb]{0.56,0.35,0.01}{\textbf{\textit{#1}}}}
\newcommand{\ErrorTok}[1]{\textcolor[rgb]{0.64,0.00,0.00}{\textbf{#1}}}
\newcommand{\ExtensionTok}[1]{#1}
\newcommand{\FloatTok}[1]{\textcolor[rgb]{0.00,0.00,0.81}{#1}}
\newcommand{\FunctionTok}[1]{\textcolor[rgb]{0.00,0.00,0.00}{#1}}
\newcommand{\ImportTok}[1]{#1}
\newcommand{\InformationTok}[1]{\textcolor[rgb]{0.56,0.35,0.01}{\textbf{\textit{#1}}}}
\newcommand{\KeywordTok}[1]{\textcolor[rgb]{0.13,0.29,0.53}{\textbf{#1}}}
\newcommand{\NormalTok}[1]{#1}
\newcommand{\OperatorTok}[1]{\textcolor[rgb]{0.81,0.36,0.00}{\textbf{#1}}}
\newcommand{\OtherTok}[1]{\textcolor[rgb]{0.56,0.35,0.01}{#1}}
\newcommand{\PreprocessorTok}[1]{\textcolor[rgb]{0.56,0.35,0.01}{\textit{#1}}}
\newcommand{\RegionMarkerTok}[1]{#1}
\newcommand{\SpecialCharTok}[1]{\textcolor[rgb]{0.00,0.00,0.00}{#1}}
\newcommand{\SpecialStringTok}[1]{\textcolor[rgb]{0.31,0.60,0.02}{#1}}
\newcommand{\StringTok}[1]{\textcolor[rgb]{0.31,0.60,0.02}{#1}}
\newcommand{\VariableTok}[1]{\textcolor[rgb]{0.00,0.00,0.00}{#1}}
\newcommand{\VerbatimStringTok}[1]{\textcolor[rgb]{0.31,0.60,0.02}{#1}}
\newcommand{\WarningTok}[1]{\textcolor[rgb]{0.56,0.35,0.01}{\textbf{\textit{#1}}}}
\usepackage{graphicx}
\makeatletter
\def\maxwidth{\ifdim\Gin@nat@width>\linewidth\linewidth\else\Gin@nat@width\fi}
\def\maxheight{\ifdim\Gin@nat@height>\textheight\textheight\else\Gin@nat@height\fi}
\makeatother
% Scale images if necessary, so that they will not overflow the page
% margins by default, and it is still possible to overwrite the defaults
% using explicit options in \includegraphics[width, height, ...]{}
\setkeys{Gin}{width=\maxwidth,height=\maxheight,keepaspectratio}
% Set default figure placement to htbp
\makeatletter
\def\fps@figure{htbp}
\makeatother
\setlength{\emergencystretch}{3em} % prevent overfull lines
\providecommand{\tightlist}{%
  \setlength{\itemsep}{0pt}\setlength{\parskip}{0pt}}
\setcounter{secnumdepth}{-\maxdimen} % remove section numbering
\ifLuaTeX
  \usepackage{selnolig}  % disable illegal ligatures
\fi

\begin{document}
\maketitle

\begin{Shaded}
\begin{Highlighting}[]
\FunctionTok{library}\NormalTok{(tidyverse)}
\end{Highlighting}
\end{Shaded}

\begin{verbatim}
## -- Attaching packages --------------------------------------- tidyverse 1.3.1 --
\end{verbatim}

\begin{verbatim}
## v ggplot2 3.3.5     v purrr   0.3.4
## v tibble  3.1.6     v dplyr   1.0.7
## v tidyr   1.1.4     v stringr 1.4.0
## v readr   2.1.1     v forcats 0.5.1
\end{verbatim}

\begin{verbatim}
## -- Conflicts ------------------------------------------ tidyverse_conflicts() --
## x dplyr::filter() masks stats::filter()
## x dplyr::lag()    masks stats::lag()
\end{verbatim}

\hypertarget{part-one-working-with-probability-density-functions}{%
\section{Part One Working with Probability Density
Functions}\label{part-one-working-with-probability-density-functions}}

\hypertarget{plot-the-density-function-for-the-t-distribution-with-44-degrees-of-freedom-or-t44.}{%
\subsection{1 Plot the density function for the t-distribution with 44
degrees of freedom, or
t(44).}\label{plot-the-density-function-for-the-t-distribution-with-44-degrees-of-freedom-or-t44.}}

\begin{Shaded}
\begin{Highlighting}[]
\FunctionTok{curve}\NormalTok{(}\FunctionTok{dt}\NormalTok{(x,}\DecValTok{44}\NormalTok{), }\AttributeTok{from =} \SpecialCharTok{{-}}\DecValTok{4}\NormalTok{, }\AttributeTok{to =} \DecValTok{4}\NormalTok{, }\AttributeTok{ylab =} \StringTok{"density"}\NormalTok{)}
\end{Highlighting}
\end{Shaded}

\includegraphics{Lab_1_files/figure-latex/unnamed-chunk-2-1.pdf}

\hypertarget{find-the-75th-percentile-for-the-t-distribution-with-44-degrees-of-freedom-df44.-report-your-answer-as-tpdfq-filling-in-values-for-pdf-and-q.}{%
\subsection{2. Find the 75th percentile for the t-distribution with 44
degrees of freedom (df=44). Report your answer as t(p;df)=q, filling in
values for p,df and
q.}\label{find-the-75th-percentile-for-the-t-distribution-with-44-degrees-of-freedom-df44.-report-your-answer-as-tpdfq-filling-in-values-for-pdf-and-q.}}

\begin{Shaded}
\begin{Highlighting}[]
\FunctionTok{qt}\NormalTok{(}\AttributeTok{p=}\NormalTok{ .}\DecValTok{75}\NormalTok{, }\AttributeTok{df=} \DecValTok{44}\NormalTok{)}
\end{Highlighting}
\end{Shaded}

\begin{verbatim}
## [1] 0.6801065
\end{verbatim}

\begin{Shaded}
\begin{Highlighting}[]
\CommentTok{\# t(.75;44)=.680}
\end{Highlighting}
\end{Shaded}

\hypertarget{consider-a-t-distribution-with-df44.-96-of-values-will-fall-between-which-two-values-of-t-that-is-find-the-values-that-satisfy-p______t44______0.96-report-both-answers-as-tpdfq.}{%
\subsection{3. Consider a t-distribution with df=44. 96\% of values will
fall between which two values of t (that is find the values that satisfy
P(\_\_\_\_\_\_\textless t(44)\textless\_\_\_\_\_\_=0.96)? Report both
answers as
t(p;df)=q.}\label{consider-a-t-distribution-with-df44.-96-of-values-will-fall-between-which-two-values-of-t-that-is-find-the-values-that-satisfy-p______t44______0.96-report-both-answers-as-tpdfq.}}

\begin{Shaded}
\begin{Highlighting}[]
\FunctionTok{qt}\NormalTok{(}\AttributeTok{p=}\NormalTok{ .}\DecValTok{02}\NormalTok{, }\AttributeTok{df=} \DecValTok{44}\NormalTok{)}
\end{Highlighting}
\end{Shaded}

\begin{verbatim}
## [1] -2.116438
\end{verbatim}

\begin{Shaded}
\begin{Highlighting}[]
\CommentTok{\# t(.02;44)={-}2.11}

\FunctionTok{qt}\NormalTok{(}\AttributeTok{p=}\NormalTok{ .}\DecValTok{98}\NormalTok{, }\AttributeTok{df=} \DecValTok{44}\NormalTok{)}
\end{Highlighting}
\end{Shaded}

\begin{verbatim}
## [1] 2.116438
\end{verbatim}

\begin{Shaded}
\begin{Highlighting}[]
\CommentTok{\#t(.98;44)=2.11}

\CommentTok{\# P({-}2.11 \textless{}t(44)\textless{} 2.11= .96)}
\end{Highlighting}
\end{Shaded}

\hypertarget{consider-a-t-distribution-with-df44.-find-pt44-1.5.}{%
\subsection{4. Consider a t-distribution with df=44. Find
P(t(44)≤-1.5).}\label{consider-a-t-distribution-with-df44.-find-pt44-1.5.}}

\begin{Shaded}
\begin{Highlighting}[]
\FunctionTok{pt}\NormalTok{(}\AttributeTok{q=}\SpecialCharTok{{-}}\FloatTok{1.5}\NormalTok{, }\AttributeTok{df=}\DecValTok{44}\NormalTok{)}
\end{Highlighting}
\end{Shaded}

\begin{verbatim}
## [1] 0.07037815
\end{verbatim}

\hypertarget{consider-a-t-distribution-with-df44.-find-pt442.}{%
\subsection{5. Consider a t-distribution with df=44. Find
P(t(44)≥2).}\label{consider-a-t-distribution-with-df44.-find-pt442.}}

\begin{Shaded}
\begin{Highlighting}[]
\FunctionTok{pt}\NormalTok{(}\AttributeTok{q=}\DecValTok{2}\NormalTok{, }\AttributeTok{df=} \DecValTok{44}\NormalTok{)}
\end{Highlighting}
\end{Shaded}

\begin{verbatim}
## [1] 0.9741517
\end{verbatim}

\hypertarget{part-2-descriptive-statistics-and-basic-graphs}{%
\section{Part 2: Descriptive Statistics and Basic
Graphs}\label{part-2-descriptive-statistics-and-basic-graphs}}

\begin{Shaded}
\begin{Highlighting}[]
\FunctionTok{getwd}\NormalTok{()}
\end{Highlighting}
\end{Shaded}

\begin{verbatim}
## [1] "/Users/brookewheeler/Desktop/Regression/Labs"
\end{verbatim}

\begin{Shaded}
\begin{Highlighting}[]
\NormalTok{nationals }\OtherTok{\textless{}{-}} \FunctionTok{read.table}\NormalTok{(}\StringTok{"../Data/nationalsdata2014.csv"}\NormalTok{, }\AttributeTok{header =} \ConstantTok{TRUE}\NormalTok{, }\AttributeTok{sep =} \StringTok{","}\NormalTok{)}

\NormalTok{nationals }\SpecialCharTok{\%\textgreater{}\%}
  \FunctionTok{select}\NormalTok{(Salary)  }\OtherTok{{-}\textgreater{}}\NormalTok{ salary}
\NormalTok{salary}
\end{Highlighting}
\end{Shaded}

\begin{verbatim}
##      Salary
## 1  20571429
## 2  14000000
## 3   8600000
## 4   7500000
## 5   7200000
## 6   6500000
## 7   6500000
## 8   5875000
## 9   5000000
## 10  3975000
## 11  3500000
## 12  3450000
## 13  3000000
## 14  2700000
## 15  2150000
## 16  2095000
## 17  1675000
## 18  1350000
## 19  1250000
## 20   950000
## 21   900000
## 22   540850
## 23   506100
## 24   504300
## 25   501400
\end{verbatim}

\hypertarget{histogram}{%
\subsection{histogram}\label{histogram}}

\begin{Shaded}
\begin{Highlighting}[]
\FunctionTok{summary}\NormalTok{(salary)}
\end{Highlighting}
\end{Shaded}

\begin{verbatim}
##      Salary        
##  Min.   :  501400  
##  1st Qu.: 1250000  
##  Median : 3000000  
##  Mean   : 4431763  
##  3rd Qu.: 6500000  
##  Max.   :20571429
\end{verbatim}

\begin{Shaded}
\begin{Highlighting}[]
\FunctionTok{hist}\NormalTok{(nationals}\SpecialCharTok{$}\NormalTok{Salary, }\AttributeTok{xlab =} \StringTok{"Salary (USD)"}\NormalTok{, }\AttributeTok{main =} \StringTok{"Histogram of 2014 Salaries"}\NormalTok{)}
\end{Highlighting}
\end{Shaded}

\includegraphics{Lab_1_files/figure-latex/unnamed-chunk-8-1.pdf}

\hypertarget{boxplot}{%
\subsection{boxplot}\label{boxplot}}

\begin{Shaded}
\begin{Highlighting}[]
\FunctionTok{boxplot}\NormalTok{(nationals}\SpecialCharTok{$}\NormalTok{Salary, }\AttributeTok{horizontal =} \ConstantTok{TRUE}\NormalTok{, }\AttributeTok{main =} \StringTok{"Boxplot of 2014 Salaries"}\NormalTok{)}
\end{Highlighting}
\end{Shaded}

\includegraphics{Lab_1_files/figure-latex/unnamed-chunk-9-1.pdf}

\end{document}
